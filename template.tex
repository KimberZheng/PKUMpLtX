%
% PKUMpLtX --- A LaTeX document class for 'Modern Physics Laboratory' in PKU based on `revtex4-2`
%
% Please read `README.md' and the template file before using
% 需要确保 font 选项指定的字体已安装! 具体参见 `README.md' 的说明.
\documentclass[font=default]{mpltx}
\usepackage{booktabs}  
% 以下至 \begin{document} 都仅是本文件为了方便额外定义的命令, 写报告时不需要.
\hypersetup{colorlinks=true}% 超链接带颜色
\usepackage{xcolor}
\newcommand{\note}[1]{{\color{gray}#1}}
\NewDocumentCommand{\pkg}{s o m}{%
    \IfBooleanF{#1}{%
        \IfNoValueTF{#2}%
            {\href{https://www.ctan.org/pkg/#3}}%
            {\href{https://www.ctan.org/pkg/#2}}%
    }%
    {\textsf{#3}}%
}
\newcommand*\cs[1]{\texttt{\textbackslash #1}}
\newcommand*\env[1]{\textit{\texttt{#1}}}
\newcommand*\code[1]{\texttt{#1}}
\newcommand*\file[1]{\textbf{\texttt{#1}}}
\makeatletter
\newcommand\releasedate{%
    \href{https://github.com/CastleStar14654/PKUMpLtX/releases/tag/\mpltx@fileversion}%
        {\mpltx@filedate, \mpltx@fileversion}}
\makeatother
% 以上是本文件为了方便额外定义的命令, 写报告时不需要.

\begin{document}
\title{He-Ne 气体激光器放电条件的研究} % 切合报告内容, 简短明确, 可以不同于讲义
\author{郑熔} % 这里 \emailphone 一定要紧跟在 \author 后方
\emailphone{2300011359@stu.pku.edu.cn}{(86)19805861588}
% 如果改用 \email 则仅需要邮箱参数
\affiliation{北京大学物理学院\quad 学号: 2300011359}
% % 可以使用 \zhdate 自动生成中文日期, 如
\date{\zhdate{2025/12/10}}
% % 也可使用 babel 的 \localedate, 如
% \date{\localedate{2020}{12}{1}}
% % 两者均会输出 `2020 年 12 月 1 日'
% 下面的 \date 的参数是为了自动输出正确版本号, 正式报告请替换为上面的两种 \date 之一
% \date{\releasedate}
\begin{abstract}
  He-Ne 气体激光器是一种常见的气体激光器, 其工作物质为氦氖混合气体. 
  它是通过电激励使氦原子与氖原子发生非弹性碰撞实现粒子数反转,从而在特定波长产生受激辐射。
  本实验研究的是632.8 nm波长的He-Ne激光器, 该波长对应于氖原子从3S到2P能级的跃迁. 通过配置一定比例的氦氖混合气体, 
  可以使用实现粒子数反转. 控制配气比不变,测量不同气体总压强下激光输出功率和放电电流的关系,从而研究He-Ne气体激光器的放电条件.
\end{abstract}
\keywords{He-Ne 气体激光器, 放电条件} % 关键词, 用逗号分隔

\maketitle
\section{引言}

激光技术的发展推动了光学、精密测量等领域的革新,而气体激光器凭借结构灵活、光谱特性优异等优势,
成为早期激光应用的核心器件之一。He-Ne 气体激光器作为典型的原子气体激光器,以氦氖混合气体为工作物质,
通过原子碰撞实现粒子数反转,最终在特定能级跃迁处产生相干辐射,其中 632.8 nm 波长的红光输出因技术成熟、稳定性高,
被广泛应用于干涉测量、准直定位等场景,是科研与工程实践中最常用的激光光源之一。
He-Ne 激光器的输出性能与其放电条件密切相关:气体配比、总压强、放电电流等参数会直接影响粒子数反转的效率与受激辐射的强度,从而影响激光输出功率。
本实验聚焦 632.8 nm 波长的 He-Ne 激光器(对应氖原子 3S \rightarrow 2P 能级跃迁),通过固定氦氖气体配比以保证粒子数反转的基础条件,
重点探究不同气体总压强下,激光输出功率与放电电流的关联特性.



\section{理论}


激光器的基本结构包括三部分:工作物质、光学谐振腔和激励能源.
要形成激光, 首先要让工作物质内部的电子在某些能级之间实现粒子数的反转分布. 粒子数反转分布的条件是
\[
\frac{g_1 N_2}{g_2 N_1} > 1. \tag{1}
\]
其中, \( N_1 \) 为下能级的粒子数密度, \( N_2 \) 为上能级的粒子数密度, \( g_1, g_2 \) 为下能级和上能级的统计权重. 
对于 He-Ne 激光器 632.8nm 谱线来说 \( g_1 = 5, g_2 = 3 \). 在 He-Ne 激光器中粒子数反转是通过气体放电来实现的.

其次, 必须满足产生激光的阈值条件, 即光在谐振腔中来回一次时在激活介质中获得的增益足以补偿各种因素导致的光的损耗. 
在忽略介质内部损耗的情况下, 阈值条件为
\[
r_1 r_2 \mathrm{e}^{2 G l} \geqslant 1. \tag{2}
\]
其中, \( r_1, r_2 \) 为谐振腔两端反射镜的反射率, \( G \) 为激活介质的增益系数, 其定义为光在单位距离内光强增加的百分比.


根据爱因斯坦受激辐射理论, 且为了满足阈值条件的要求, 在忽略介质内部损耗的情况下, 增益系数应当至少有
\[
G_{\text{min}} = \frac{1}{2l} \ln (r_1 r_2)^{-1} 
\]


He-Ne 激光器 632.8nm 谱线激光对应的上能级为 Ne 的 \( ^3\mathrm{S}_2 \) 态, 
下能级为 \( ^2\mathrm{P}_4 \) 态, 
632.8nm 的激光是这两个能级之间实现粒子数反转而产生的.


在混合气体中,  He 的激发态 \( ^1\mathrm{S}_0 \) 与 Ne 的激发态 \( ^3\mathrm{S}_2 \) 之间的能量非常接近, 
在合适的配气比的情况下, 由于 He 的 \( ^1\mathrm{S}_0 \) 态是亚稳态, 它们可以与基态 Ne 原子碰撞而发生能量的共振转移, 
把基态的 Ne 原子激发到 \( ^3\mathrm{S}_2 \) 能级上去而自己回到 He 的基态.
由于这两个能级的能量很接近, 发生上述能量共振转移的截面很大, 使上能级的粒子数密度较大.


另一方面, 对于下能级 \( ^2\mathrm{P}_4 \) 态来说, 在偶极辐射近似下它与基态之间属于禁跃迁, 
并且电子碰撞使得 Ne 由基态激发到 \( ^2\mathrm{P}_4 \) 态的碰撞截面也很小, 而这一态的寿命也很短, 
因此下能级的粒子数密度很小. 因此很容易实现 Ne 的 \( ^3\mathrm{S}_2 \) 和 \( ^2\mathrm{P}_4 \) 能级之间的粒子数反转分布.

根据大量实验可以总结出以下规律:
气体配比最好为 \( p_{\text{He}}: p_{\text{Ne}} = 5: 1 \sim  7 : 1\). 
根据毛细管直径 \( d \) 的大小, 由公式 \( pd = 300 \sim 500 \, 
\text{Pa·mm} \) 可求出最佳工作总压强在 300 Pa 左右. 
对于本实验中毛细管直径 \( d \) 为 1.25 mm 的 He-Ne 激光器, 最佳充气总压强在 300 Pa 左右.


\section{实验}
\subsection{实验装置示意}
实验装置如\autoref{fig:zhuangzhi}所示. 

\begin{figure}[htbp]
  \centering
  \includegraphics[width=0.6\linewidth]{fig/zhuangzhi.png}
  \caption{He-Ne 气体激光器实验装置示意图\cite{jindaiwulishiyan}}
  \label{fig:zhuangzhi}
\end{figure}

本实验实验装置如\autoref{fig:zhuangzhi}所示, 
主要由激光管、测量系统和真空系统组成.
实验通过真空系统进行抽气、配气, 控制 He-Ne 气体配比与激光管内气体压强, 
测量统系通过利用激光管中的电极进行放电, 采集 He-Ne 激光电源的放电电流和利用光功率计测定的激光器输出光功率.


扩散泵用于高真空抽气, 可以让气压降到$\sim 10^{-3}Pa$。控制双通阀和三通阀可以控制气体的流向,从而实现抽气和配气.

测量系统中,U型压力计中油的密度$\rho = 1.09 g/cm^3$, 重力加速度$g = 9.8 m/s^2$, 故 1 mm 的压强差对应油柱高度差为$ 10.682 Pa$.



\subsection{实验步骤}
  \subsubsection{抽空除气}
  先用机械泵,抽至$1 Pa$以下,再打开扩散泵和真空计,将气压抽至$10^{-3} Pa$左右.



  \subsubsection{配气}
  由于本实验装置中 He 气的阀有少许漏气,因此采用Ne气度量$V_1$和$V_2$的体积比($V_1$是指两个气体气瓶部分的体积,而$V_2$是指激光管部分的体积).
  在$V_1$中冲入Ne气,测得U型压力计的高度差为$\Delta h = 23.73 cm$,打开连接$V_1$和$V_2$的阀门,待压力平衡后测得U型压力计的高度差为$\Delta h = 6.5 cm$,则
  可以得到$V_1$和$V_2$的体积比为:
  \[
  \frac{V_1}{V_2} = \frac{6.5}{23.73-6.5} = 0.377
  \]
  将$V_1$与$V_2$抽空后,向$V_1$与$V_2$中充入Ne气,测得U型压力计的高度差为$\Delta h_{{Ne}}' = 12 mm$。
  然后抽走$V_1$中的Ne气,向$V_1$中充入He气,测得U型压力计的高度差为$\Delta h_{{He}}' = 23 cm$.
  考虑混合后 \( p_{\text{He}}/p_{\text{Ne}} = \eta \), \( p_{\text{He}} + p_{\text{Ne}} = p_{\text{tot}} \),根据 \textbf{式 (5)},我们能够得到
  \[
  \left\{
  \begin{aligned}
  &p_{\text{He}}' V_1 = p_{\text{He}} (V_1 + V_2), \\
  &p_{\text{Ne}}' V_2 = p_{\text{Ne}} (V_1 + V_2),
  \end{aligned}
  \right.
  \]


  所以, 可以计算得到 \( \eta = p_{\text{He}}'V_1/p_{\text{Ne}}'V_2 = 7.2 \approx 7:1 \).
  打开连接$V_1$和$V_2$的阀门,待压力平衡后测得U型压力计的高度差为$\Delta h = 7.6 cm$,则激光管内的总压强为: $$p_{\text{tot}} = \rho g \Delta h = 1.09 \times 10^3  \times 9.8  \times 0.076 m = 812 Pa.$$
  \subsubsection{测量放电条件对激光输出功率的影响}

  保持气体配比不变,调节总压强$p_{\text{tot}}$,测量不同气体总压强下激光输出功率和放电电流的关系. 注意电流超过 10 mA 的时间不宜过长。



\section{结果及讨论}
    按照上述实验步骤对仪器进行调节, 得到起偏器的刻度为\(339^\circ45'\),光弹调制器的角度为\(280^\circ0'\), 检偏器的角度为\(43^\circ45'\).

    克尔转角的标度系数为\(-6.84\times10^{-2}/^\circ\). 

  \subsection{测量克尔磁滞回线}

    采用起始磁场为-1000 mT, 末态磁场为1000 mT, 步长为200 mT, 对起偏器的刻度为\(339^\circ45'\)下的克尔磁滞回线进行测量, 得到克尔转角和克尔椭率. 但由于该图像的对称性较差, 对于饱和克尔转角、矫顽力的读数、以及后续的分析不太友好, 
    我们选取相对起偏器的刻度为\(339^\circ45'\)转动\(-1^\circ\)的图像进行分析, 其克尔转角和克尔椭率分别如\autoref{fig:keerzhuanjiao}和\autoref{fig:keertuolv}所示,该图对于实验中采集得到的数据点做了线性内插. 

    \begin{figure}[htbp]
      \centering
      \includegraphics[width=0.6\linewidth]{fig/new2.png}
      \caption{克尔转角磁滞回线}
      \label{fig:keerzhuanjiao}
    \end{figure}

    \begin{figure}[htbp]
      \centering
      \includegraphics[width=0.6\linewidth]{fig/new3.png}
      \caption{克尔椭率磁滞回线}
      \label{fig:keertuolv}

    \end{figure}

      观察克尔转角的磁滞回线并测量, 可以发现: 
      \begin{enumerate}
        \item 克尔转角具有较为明显的磁滞回线特征. 
        \item 饱和克尔转角为\(\theta_{ks} = 0.196^\circ\)
        \item 矫顽力为\(B = 339.432 mT\)
      \end{enumerate}

      观察克尔椭率的磁滞回线, 发现: 
      \begin{enumerate}
        \item  克尔椭率的磁滞回线特征并不明显. 
        \item 这是因为, 克尔椭率主要由一次谐波分量决定, 而实验装置中, 光线容易被反射、折射, 导致与我们希望测量的一次谐波分量产生干涉. 从而导致结果较为混乱. 
      \end{enumerate}

  \subsection{探究克尔转角和起偏器角度的关系}
      根据实验步骤所示多次改变起偏器的角度, 得到磁滞回线如\autoref{fig:keerzhuanjiao2}所示.

      \begin{figure}[htbp]
        \centering
        \includegraphics[width=0.6\linewidth]{fig/new1.png}
        \caption{不同起偏器角度下的克尔转角磁滞回线}
        \label{fig:keerzhuanjiao2}
      \end{figure}
      
      分析上述图像, 
      可以发现各条曲线磁滞回线的形态无显著变化, 相互之间近似为上下平移的关系. 
      读取每条磁滞回线的中心克尔转角与起偏器转角之间的关系, 得到如\autoref{fig:guifan}所示的关系图.
      \begin{figure}[htbp]
        \centering
        \includegraphics[width=0.6\linewidth]{fig/nihe.png}
        \caption{起偏器转角与中心克尔转角的关系}
        \label{fig:guifan}
      \end{figure}

      可以发现中心克尔转角与起偏器转角呈线性关系, 斜率为0.816, 与1非常接近, 验证了测量的克尔转角与起偏器夹角之间为\(\theta_k = \theta_0 + \theta_{ki}\)的关系. 
      其中, \(\theta_k\)为测量的克尔转角, \(\theta_0\)为某一常数, \(\theta_{ki}\)为样品本身的克尔转角. 

      

\section{结论}

  本次实验, 我们利用样品\(Pt_{73}Co_{27}\) 合金薄膜, 借助光弹调制器、锁相放大器等, 通过椭偏检测技术实现了对微小克尔转角的测量, 观测到了磁光克尔效应. 
  通过小范围下改变起偏器角度, 确定标定系数为\(-6.84\times10^{-2}/^\circ\). 
  通过测定样品的克尔磁滞回线, 得到了样品对于$632.8\mathrm{nm}$波长的激光的饱和克尔转角$\theta_{ks}=0.196^\circ$, 矫顽力为$B_c=339.432mT$. 
  证明了样品具有铁磁性. 
  同时研究了起偏器角度对于克尔转角的影响, 验证了测量的克尔转角与起偏器夹角之间为线性关系. 

\begin{acknowledgments}
  感谢搭档吴浅溪和周路群老师的指导和帮助.
\end{acknowledgments}

% bibliography 的参数是你的 *.bib 文件去掉后缀名后的部分
\bibliography{bibli}

\appendix

\section{思考题}
\subsection{我们的实验装置对克尔转角和克尔椭偏率的测量精度是否一样高?为什么?}
  不一样高.克尔转角正比于二次谐波分量, 克尔椭偏率正比于一次谐波分量.
  光束在光弹调制晶体的两个表面和其他地方发生多次折射、反射, 导致光
  束干涉现象, 影响一次谐波的测量, 所以对克尔椭偏率的测量是不准确的.

\subsection{如果用一个以角速度$\omega$旋转的$\frac{\lambda}{2}$波片代替光弹调制器, 光电探测器的输出信号会如何变化, 是否也能测出复克尔转角?}
以角速度$\omega$转动的$\frac{\lambda}{2}$波片, 其琼斯矩阵为: 
\[
\begin{bmatrix}
\cos\omega t & -\sin\omega t \\
\sin\omega t & \cos\omega t
\end{bmatrix}
\cdot
\begin{bmatrix}
1 & 0 \\
0 & -1
\end{bmatrix}
\cdot
\begin{bmatrix}
\cos\omega t & \sin\omega t \\
-\sin\omega t & \cos\omega t
\end{bmatrix}
=
\begin{bmatrix}
\cos2\omega t & \sin2\omega t \\
\sin2\omega t & -\cos2\omega t
\end{bmatrix}
\]


\[
\begin{bmatrix}
\frac{1}{2} & \frac{1}{2} \\
\frac{1}{2} & \frac{1}{2}
\end{bmatrix}
\begin{bmatrix}
\cos2\omega t & \sin2\omega t \\
\sin2\omega t & -\cos2\omega t
\end{bmatrix}
\begin{bmatrix}
r_F \\
k
\end{bmatrix}
=
\frac{1}{2}
\begin{bmatrix}
(\sin2\omega t + \cos2\omega t)r_F + (\sin2\omega t - \cos2\omega t)k \\
(\sin2\omega t + \cos2\omega t)r_F + (\sin2\omega t - \cos2\omega t)k
\end{bmatrix}
\]

光强为: 
\[
I(t) = r_F^2 + k^2 + r_F^2 \sin4\omega t - k^2 \sin4\omega t + r_F k (\sin 4 \omega t - 1).
\]


由此可见,  出射光的强度包含直流分量和四次谐波分量,  而四次谐波分量中耦合信息, 
因此无法实现对复克尔转角的实部和
虚部的有效提取,  因此不能测出复克尔转角. 
\end{document}
