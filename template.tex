%
% PKUMpLtX --- A LaTeX document class for 'Modern Physics Laboratory' in PKU based on `revtex4-2`
%
% Please read `README.md' and the template file before using
% 需要确保 font 选项指定的字体已安装! 具体参见 `README.md' 的说明.
\documentclass[font=default]{mpltx}
\usepackage{booktabs}  
% 以下至 \begin{document} 都仅是本文件为了方便额外定义的命令, 写报告时不需要.
\hypersetup{colorlinks=true}% 超链接带颜色
\usepackage{xcolor}
\newcommand{\note}[1]{{\color{gray}#1}}
\NewDocumentCommand{\pkg}{s o m}{%
    \IfBooleanF{#1}{%
        \IfNoValueTF{#2}%
            {\href{https://www.ctan.org/pkg/#3}}%
            {\href{https://www.ctan.org/pkg/#2}}%
    }%
    {\textsf{#3}}%
}
\newcommand*\cs[1]{\texttt{\textbackslash #1}}
\newcommand*\env[1]{\textit{\texttt{#1}}}
\newcommand*\code[1]{\texttt{#1}}
\newcommand*\file[1]{\textbf{\texttt{#1}}}
\makeatletter
\newcommand\releasedate{%
    \href{https://github.com/CastleStar14654/PKUMpLtX/releases/tag/\mpltx@fileversion}%
        {\mpltx@filedate, \mpltx@fileversion}}
\makeatother
% 以上是本文件为了方便额外定义的命令, 写报告时不需要.

\begin{document}
\title{He-Ne 气体激光器放电条件的研究} % 切合报告内容, 简短明确, 可以不同于讲义
\author{郑熔} % 这里 \emailphone 一定要紧跟在 \author 后方
\emailphone{2300011359@stu.pku.edu.cn}{(86)19805861588}
% 如果改用 \email 则仅需要邮箱参数
\affiliation{北京大学物理学院\quad 学号: 2300011359}
% % 可以使用 \zhdate 自动生成中文日期, 如
\date{\zhdate{2025/12/10}}
% % 也可使用 babel 的 \localedate, 如
% \date{\localedate{2020}{12}{1}}
% % 两者均会输出 `2020 年 12 月 1 日'
% 下面的 \date 的参数是为了自动输出正确版本号, 正式报告请替换为上面的两种 \date 之一
% \date{\releasedate}
\begin{abstract}
  He-Ne 气体激光器是一种常见的气体激光器, 其工作物质为氦氖混合气体. 
  它是通过电激励使氦原子与氖原子发生非弹性碰撞实现粒子数反转, 从而在特定波长产生受激辐射. 
  本实验研究的是632.8 nm波长的He-Ne激光器, 该波长对应于氖原子从3S到2P能级的跃迁. 通过配置一定比例的氦氖混合气体, 
  可以使用实现粒子数反转. 控制配气比不变, 测量不同气体总压强下激光输出功率和放电电流的关系, 从而研究He-Ne气体激光器的放电条件.
\end{abstract}
\keywords{He-Ne 气体激光器, 放电条件} % 关键词, 用逗号分隔

\maketitle
\section{引言}

激光技术的发展推动了光学、精密测量等领域的革新, 而气体激光器凭借结构灵活、光谱特性优异等优势, 
成为早期激光应用的核心器件之一. He-Ne 气体激光器作为典型的原子气体激光器, 以氦氖混合气体为工作物质, 
通过原子碰撞实现粒子数反转, 最终在特定能级跃迁处产生相干辐射, 其中 632.8 nm 波长的红光输出因技术成熟、稳定性高, 
被广泛应用于干涉测量、准直定位等场景, 是科研与工程实践中最常用的激光光源之一. 
He-Ne 激光器的输出性能与其放电条件密切相关: 气体配比、总压强、放电电流等参数会直接影响粒子数反转的效率与受激辐射的强度, 从而影响激光输出功率. 
本实验聚焦 632.8 nm 波长的 He-Ne 激光器(对应氖原子 3S \rightarrow 2P 能级跃迁), 通过固定氦氖气体配比以保证粒子数反转的基础条件, 
重点探究不同气体总压强下, 激光输出功率与放电电流的关联特性.



\section{理论}


激光器的基本结构如\autoref{fig:laser}所示, 包括三部分: 工作物质、光学谐振腔和激励能源.
\begin{figure}
  \centering
  \includegraphics[width=0.8\linewidth]{fig/jiguang.png.png}
  \caption{激光器基本结构示意图\cite{jindaiwulishiyan}}
  \label{fig:laser}
\end{figure}
要形成激光, 首先要让工作物质内部的电子在某些能级之间实现粒子数的反转分布. 粒子数反转分布的条件是
\[
\frac{g_1 N_2}{g_2 N_1} > 1. \tag{1}
\]
其中, \( N_1 \) 为下能级的粒子数密度, \( N_2 \) 为上能级的粒子数密度, \( g_1, g_2 \) 为下能级和上能级的统计权重. 
对于 He-Ne 激光器 632.8nm 谱线来说 \( g_1 = 5, g_2 = 3 \). 在 He-Ne 激光器中粒子数反转是通过气体放电来实现的.

其次, 必须满足产生激光的阈值条件, 即光在谐振腔中来回一次时在激活介质中获得的增益足以补偿各种因素导致的光的损耗. 
在忽略介质内部损耗的情况下, 阈值条件为
\[
r_1 r_2 \mathrm{e}^{2 G l} \geqslant 1. \tag{2}
\]
其中, \( r_1, r_2 \) 为谐振腔两端反射镜的反射率, \( G \) 为激活介质的增益系数, 其定义为光在单位距离内光强增加的百分比.


根据爱因斯坦受激辐射理论, 且为了满足阈值条件的要求, 在忽略介质内部损耗的情况下, 增益系数应当至少有
\[
G_{\text{min}} = \frac{1}{2l} \ln (r_1 r_2)^{-1} 
\]


He-Ne 激光器 632.8nm 谱线激光对应的上能级为 Ne 的 \( ^3\mathrm{S}_2 \) 态, 
下能级为 \( ^2\mathrm{P}_4 \) 态, 
632.8nm 的激光是这两个能级之间实现粒子数反转而产生的.


在混合气体中,  He 的激发态 \( ^1\mathrm{S}_0 \) 与 Ne 的激发态 \( ^3\mathrm{S}_2 \) 之间的能量非常接近, 
在合适的配气比的情况下, 由于 He 的 \( ^1\mathrm{S}_0 \) 态是亚稳态, 它们可以与基态 Ne 原子碰撞而发生能量的共振转移, 
把基态的 Ne 原子激发到 \( ^3\mathrm{S}_2 \) 能级上去而自己回到 He 的基态.
由于这两个能级的能量很接近, 发生上述能量共振转移的截面很大, 使上能级的粒子数密度较大.


另一方面, 对于下能级 \( ^2\mathrm{P}_4 \) 态来说, 在偶极辐射近似下它与基态之间属于禁跃迁, 
并且电子碰撞使得 Ne 由基态激发到 \( ^2\mathrm{P}_4 \) 态的碰撞截面也很小, 而这一态的寿命也很短, 
因此下能级的粒子数密度很小. 因此很容易实现 Ne 的 \( ^3\mathrm{S}_2 \) 和 \( ^2\mathrm{P}_4 \) 能级之间的粒子数反转分布.

根据大量实验可以总结出以下规律: 
气体配比最好为 \( p_{\text{He}}: p_{\text{Ne}} = 5: 1 \sim  7 : 1\). 
根据毛细管直径 \( d \) 的大小, 由公式 \( pd = 300 \sim 500 \, 
\text{Pa·mm} \) 可求出最佳工作总压强在 300 Pa 左右. \cite{jindaiwulishiyan}


\section{实验}
\subsection{实验装置示意}

\begin{figure}[htbp]
  \centering
  \includegraphics[width=0.8\linewidth]{fig/zhuangzhi.png}
  \caption{He-Ne 气体激光器实验装置示意图\cite{jindaiwulishiyan}}
  \label{fig:zhuangzhi}
\end{figure}

本实验实验装置如\autoref{fig:zhuangzhi}所示, 
主要由激光管、测量系统和真空系统组成.
实验通过真空系统进行抽气、配气, 控制 He-Ne 气体配比与激光管内气体压强, 
测量统系通过利用激光管中的电极进行放电, 采集 He-Ne 激光电源的放电电流和利用光功率计测定的激光器输出光功率.


扩散泵用于高真空抽气, 可以让气压降到$\sim 10^{-3}Pa$. 控制双通阀和三通阀可以控制气体的流向,从而实现抽气和配气.

测量系统中, U型压力计中油的密度$\rho = 1.09 g/cm^3$, 重力加速度$g = 9.8 m/s^2$, 故 1 mm 的压强差对应油柱高度差为$ 10.682 Pa$.



\subsection{实验步骤}
  \subsubsection{抽空除气}
  先用机械泵, 抽至$1 Pa$以下, 再打开扩散泵和真空计, 将气压抽至$10^{-3} Pa$左右.



  \subsubsection{配气}
  由于本实验装置中 He 的阀有少许漏气, 因此采用Ne度量$V_1$和$V_2$的体积比($V_1$是指两个气体气瓶部分的体积, 而$V_2$是指激光管部分的体积).
  在$V_1$中冲入Ne, 测得U型压力计的高度差为$\Delta h = 23.73 cm$, 打开连接$V_1$和$V_2$的阀门, 待压力平衡后测得U型压力计的高度差为$\Delta h = 6.5 cm$, 则
  可以得到$V_1$和$V_2$的体积比为:
  \[
  \frac{V_1}{V_2} = \frac{6.5}{23.73-6.5} = 0.377
  \]
  将$V_1$与$V_2$抽空后, 向$V_1$与$V_2$中充入Ne, 测得U型压力计的高度差为$\Delta h_{{Ne}}' = 12 mm$. 
  然后抽走$V_1$中的Ne, 向$V_1$中充入He, 测得U型压力计的高度差为$\Delta h_{{He}}' = 23 cm$.
  考虑混合后 \( p_{\text{He}}/p_{\text{Ne}} = \eta \), \( p_{\text{He}} + p_{\text{Ne}} = p_{\text{tot}} \), 根据 \textbf{式 (5)}, 我们能够得到
  \[
  \left\{
  \begin{aligned}
  &p_{\text{He}}' V_1 = p_{\text{He}} (V_1 + V_2), \\
  &p_{\text{Ne}}' V_2 = p_{\text{Ne}} (V_1 + V_2),
  \end{aligned}
  \right.
  \]


  所以, 可以计算得到 \( \eta = p_{\text{He}}'V_1/p_{\text{Ne}}'V_2 = 7.2 \approx 7:1 \).
  打开连接$V_1$和$V_2$的阀门, 待压力平衡后测得U型压力计的高度差为$\Delta h = 7.6 cm$, 则激光管内的总压强为: $$p_{\text{tot}} = \rho g \Delta h = 1.09 \times 10^3  \times 9.8  \times 0.076 m = 812 Pa.$$
  \subsubsection{测量放电条件对激光输出功率的影响}

  保持气体配比不变, 调节总压强$p_{\text{tot}}$, 测量不同气体总压强下激光输出功率和放电电流的关系. 注意电流超过 10 mA 的时间不宜过长. 



\section{结果及讨论}

  根据上述实验步骤, 在 \( p_{\text{He}} : p_{\text{Ne}} = 7.2 : 1\)的前提下, 我们测得了不同总压强下的放电电流和激光输出功率, 结果如\autoref{fig:plot}所示.

  \begin{figure}[htbp]
    \centering
    \includegraphics[width=0.8\linewidth]{fig/plot.png}
    \caption{不同总压强下的放电电流和激光输出功率关系图}
    \label{fig:plot}

  \end{figure}

  可以发现, 在压强为 320.46 Pa时, 激光输出功率随着放电电流先增加后减小, 在电流为8mA左右时达到峰值. 


  出现这种现象的原因可能是: 放电电流的大小会对电子碰撞的强度有影响, 
  电流较小的时候, 亚稳态 $2^1\mathrm{S}_0$ 的 $\mathrm{He}$ 的原子密度随电流逐渐增大;而当电流增加到一定程度时, 
  亚稳态 $2^1\mathrm{S}_0$ 的 $\mathrm{He}$ 的原子密度有饱和的趋势. 
  随放电电流的增加, 电子与 $\mathrm{Ne}$ 上能级原子碰撞并激发到下能级的概率也增加, 抑制粒子数的反转. 
  在二者的共同影响下, 在某一电流值时, 粒子数反转达到最大值. 


  根据上面的分析, 理论上在一定总压强范围内, 固定总压强时, 激光输出功率都会随着放电电流的增加先增大后减小. 
  但是实验结果\autoref{fig:plot}中, 并没有体现这一点. 这可能是因为考虑到放电电流最好不要长时间超过 10 mA, 因此实验中的取样点较少, 未能覆盖到所有的放电电流范围. 
  但我们可以合理外推, 在其他总压强下, 激光输出功率也会随着放电电流的增加先增大后减小, 并且达到功率峰值的电流会随着总压强的减小而增大. 


  产生这一现象的原因是随着总压强的减小, 气体分子密度降低, 电子与气体分子碰撞的频率降低, 
  需要更大的放电电流来维持足够的碰撞频率, 从而实现粒子数反转. 


  此外, 分析\autoref{fig:plot}可以发现, 在总压强很高(如811.83 Pa)时, 激光输出功率较低, 这可能是因为高压强下, 气体分子密度过大, 
  导致电子的自由程较短, 电子碰撞频率较大, 在碰撞过程中能量损失较大, 难以有效激发氖原子, 从而降低了粒子数反转的效率, 进而影响激光输出功率. 
  
  综合来看, 高压强和低压强都会对激光输出功率产生不利影响, 存在一个最佳的激光放电的总压强范围. 





\section{结论}

  本次实验, 我们研究了 He-Ne 气体激光器在不同放电条件下的输出功率特性. 采用自行设置的配气方案, 在配气比例为 \( p_{\text{He}} : p_{\text{Ne}} = 7.2 : 1\) 的前提下,
  我们测量了不同总压强下激光输出功率与放电电流的关系. 实验结果表明, 在一定压强范围内, 激光输出功率随着放电电流的增加先增大后减小, 
  并且在不同总压强下达到峰值的放电电流有所不同. 且高压强和低压强都会降低激光输出功率, 存在一个最佳的激光放电的总压强范围.

\begin{acknowledgments}
  感谢搭档和彭丽聪老师的指导和帮助.
\end{acknowledgments}

% bibliography 的参数是你的 *.bib 文件去掉后缀名后的部分
\bibliography{bibli}

\appendix

\section{思考题}
\subsection{不同的总压强下, 激光输出功率的峰值的位置是如何变化的?如何理解?}
  在一定压强范围内, 随着总压强的减小, 激光输出功率的峰值位置对应的放电电流会增大, 峰值会先上升后下降降低. 
  这是因为随着总压强的减小, 气体分子密度降低, 电子与气体分子碰撞的频率降低, 
  需要更大的放电电流来维持足够的碰撞频率, 从而实现粒子数反转. 
  而峰值先上升后下降是因为高压强下, 气体分子密度过大, 
  导致电子的自由程较短, 电子碰撞频率较大, 在碰撞过程中能量损失较大, 难以有效激发氖原子, 从而降低了粒子数反转的效率, 进而影响激光输出功率. 
  而低压强下, 气体分子密度过低, 电子与气体分子碰撞的频率过低, 也不利于粒子数反转的实现. 

\subsection{实验中, 当 He-Ne 气体压强和放电电流保持不变时, 激光器光输出功率存在小的起伏, 原因何在?}
  这是因为实验中激光器的放电条件并非完全稳定, 可能存在微小的波动, 例如电子密度的波动, 气流存在扰动等, 都会影响激光输出功率的稳定性. 
  并且, 实验测量过程中可能存在环境光干扰, 测量仪器的灵敏度和响应时间等因素也会引入一定的误差, 导致测量结果出现小的起伏. 
  此外, 激光器本身的热效应也可能导致输出功率的波动, 当激光器工作一段时间后, 温度升高可能会影响气体的密度和放电特性, 从而引起输出功率的变化. 




\end{document}
