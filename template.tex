%
% PKUMpLtX --- A LaTeX document class for 'Modern Physics Laboratory' in PKU based on `revtex4-2`
%
% Please read `README.md' and the template file before using
% 需要确保 font 选项指定的字体已安装! 具体参见 `README.md' 的说明.
\documentclass[font=default]{mpltx}
\usepackage{booktabs}  
% 以下至 \begin{document} 都仅是本文件为了方便额外定义的命令, 写报告时不需要.
\hypersetup{colorlinks=true}% 超链接带颜色
\usepackage{xcolor}
\newcommand{\note}[1]{{\color{gray}#1}}
\NewDocumentCommand{\pkg}{s o m}{%
    \IfBooleanF{#1}{%
        \IfNoValueTF{#2}%
            {\href{https://www.ctan.org/pkg/#3}}%
            {\href{https://www.ctan.org/pkg/#2}}%
    }%
    {\textsf{#3}}%
}
\newcommand*\cs[1]{\texttt{\textbackslash #1}}
\newcommand*\env[1]{\textit{\texttt{#1}}}
\newcommand*\code[1]{\texttt{#1}}
\newcommand*\file[1]{\textbf{\texttt{#1}}}
\makeatletter
\newcommand\releasedate{%
    \href{https://github.com/CastleStar14654/PKUMpLtX/releases/tag/\mpltx@fileversion}%
        {\mpltx@filedate, \mpltx@fileversion}}
\makeatother
% 以上是本文件为了方便额外定义的命令, 写报告时不需要.

\begin{document}

\title{核磁共振} % 切合报告内容, 简短明确, 可以不同于讲义
\author{郑熔} % 这里 \emailphone 一定要紧跟在 \author 后方
\emailphone{2300011359@stu.pku.edu.cn}{(86)19805861588}
% 如果改用 \email 则仅需要邮箱参数
\affiliation{北京大学物理学院\quad 学号: 2300011359}
% % 可以使用 \zhdate 自动生成中文日期, 如
\date{\zhdate{2025/11/12}}
% % 也可使用 babel 的 \localedate, 如
% \date{\localedate{2020}{12}{1}}
% % 两者均会输出 `2020 年 12 月 1 日'
% 下面的 \date 的参数是为了自动输出正确版本号, 正式报告请替换为上面的两种 \date 之一
% \date{\releasedate}
\begin{abstract}
  核磁共振是具有非零自旋的原子核在外磁场作用下,
  吸收特定频率电磁波发生能级跃迁的物理过程。
  本实验从共振频率的基础调节操作入手,
  首先观察了掺有三氯化铁的水样品的核磁共振基本现象,探究射频场频率、幅度、扫场幅度以及电路盒左右位置对共振信号的波形、幅度和位置分布的影响。
  随后,利用共振条件以及水样品中质子的共振频率对永磁铁的磁场强度进行校准,并基于校准值测量聚四氟乙烯样品中氟原子核的 g 因子,
  同时测定氟核的横向弛豫时间。
\end{abstract}
\keywords{核磁共振,g 因子,弛豫时间} % 关键词, 用逗号分隔

\maketitle
\section{引言}
Pauli于1924年研究元素光谱精细结构时首次提出核磁矩与核自旋概念。
1939年,经I.I.Rabi改进的Otto Stern首创分子束实验,
提出更精确的核磁矩测量方法。
1946年,美国Harvard大学的Purcell、Pound与Stanford大学的Bloch、Hanson同时独立设计近代核磁共振技术。
在核物理领域,通过测量各种核磁矩大小,可获取诸多核结构信息。
本实验旨探究了核磁共振实验现象,并利用核磁共振校准磁场,测量氟核和\( g \)因子以及横向弛豫时间。

\section{理论}
根据共振时 \( h\nu = \Delta E \) 和\( \Delta E = \gamma \hbar B\)可以把共振条件写成
\[
\omega = \gamma B \quad \text{或} \quad \nu = \frac{\gamma}{2\pi} B 
\]
式中 \( \omega = 2\pi\nu \) 为射频场的圆频率.
\( \gamma/2\pi \) 称为原子核的回旋频率,数值上等于磁场为1 T时的磁共振频率,
单位为 MHz/T.若已知磁场 \( B \),测量共振频率可求出 \( \gamma/2\pi \) 的数值.反之,可利用\( \gamma/2\pi \) 为已知值来校准磁场。



热平衡条件下之所以可以观测到核磁共振现象,是因为体系上下能级的粒子数分布遵循玻尔兹曼分布:
\[
\frac{N_{20}}{N_{10}} = \exp(-\Delta E/kT)
\]
其中 \( N_{20} \)、\( N_{10} \) 分别是上、下能级的粒子数。一般情形下,\( \Delta E \ll kT \),近似有
\[
\frac{N_{20}}{N_{10}} \approx 1 - \frac{\Delta E}{kT} = 1 - \frac{\gamma \hbar B}{kT}
\]
由此可求出热平衡时下能级与上能级的粒子数之差为
\[
n_0 = N_{10} - N_{20} \approx \frac{\gamma \hbar B}{2kT}N \tag{6-1-4}
\]
其中 \( N \) 为粒子总数,对于氢核,在室温下当磁场为 \( 1 \, \text{T} \) 时
\[
n_0 \approx 0.000\ 003\ 4\ N
\]
这个粒子数差提供了观察核磁共振的可能性。
磁场 \( B \) 越强,粒子差数越大,对观察核磁共振信号越有利;而温度越高,粒子差数越小,对观察核磁共振信号越不利。

粒子的横向弛豫时间由以下公式决定:
\[
\frac{1}{T_2} = \frac{\Delta \omega}{2} 
\]

其中 \( \Delta \omega \) 为共振峰的半高宽。

示波器产生的波形与扫场速度、横向弛豫时间有关\autoref{saochang}。扫场速度越快,尾波数量越多。外磁场越均匀,横向弛豫时间越长,尾波数量越多。以此可以判断外磁场均匀性。

\begin{figure}[htbp]
  \centering
  \includegraphics[width=0.6\linewidth]{fig/saochang.png}
  \caption{扫场速度不同时所观测得到的示波器信号波形\cite{jindaiwulishiyan}}
  \label{fig:saochang}
\end{figure}

\section{实验}
\subsection{实验装置示意}
实验装置如\autoref{fig:zhaungzhi}所示. 

\begin{figure}[htbp]
  \centering
  \includegraphics[width=0.6\linewidth]{fig/shiyanzhaungzhi.png}
  \caption{核磁共振实验装置示意图\cite{jindaiwulishiyan}}
  \label{fig:zhaungzhi}
\end{figure}


\subsection{实验步骤}
  \subsubsection{观察掺有三氯化铁水样品的核磁共振现象,确定均匀场的位置}
    将水样品(含三氯化铁)放置在核磁共振仪中,
    将扫场线圈的幅度置于 100 格左右,将 “检波输出” 端和示波器连接,
    将 “频率测试” 端和数字频率计连接,调节示波器内扫描为 5ms / 格。
    连续调节边限振荡器的频率,并观察示波器波形,直到可以观察到共振吸收峰。
    之后依次改变 \textbf {射频场频率}、\textbf {扫场幅度}、\textbf {电路盒左右位置} 和 \textbf {射频场幅度},
    观察对吸收信号的波形、幅度以及位置分布的影响。


    调整电路盒的左右位置,同时观察示波器上的波形,当波形中尾波最多时,记录下此时电路盒的位置,这时候外磁场的均匀性最好。

  \subsubsection{校准永磁铁的磁场}
    将样品置于前面中所测量的磁场均匀性最好的位置,
    找到共振吸收信号后尽可能减小扫场幅度,这样可以减小共振频率测量误差。
    调整射频场频率使得示波器上出现多个共振峰,峰与峰之间间隔为 10ms 左右,记录下此时的射频场频率\( \nu_H \)。
    根据共振条件和\( 25^\circ\text{C} \)水样品中质子的回旋频率\( \frac{\gamma}{2\pi} \),
    可以求出磁场的大小
    \[
    B_0 = \frac{\nu_H}{\gamma/2\pi}
    \]

    为了估计频率测量的精度,可以调节射频场频率使得示波器每两个共振峰合并为1个,即峰与峰之间间隔变为 20ms 左右,记录下满足该情形的频率分别为 \( \nu_H' \) 和 \( \nu_H'' \)( \( \nu_H' < \nu_H'' \))。
    由此对频率测量的不确定度进行估计
    \[
    \Delta B = \frac{B'}{10} = \frac{(\nu_H' - \nu_H'')/20}{\gamma/2\pi}
    \]
    从而可以得到磁场的不确定度。

    \subsection{测量氟核的 \( g \) 因子}

      将聚四氟乙烯样品盒放在与2中相同的位置(均匀磁场处),调节共振后利用
      \[
      g = \frac{\nu_F / B_0}{\mu_N / h}
      \]
      计算氟核的 \( g \) 因子,可以利用和2中相同的方法测量出频率的精度,并结合磁场的精度对结果的不确定度进行估计。

    \subsection{测量氟核的横向弛豫时间}

      将“扫场输出”信号作为示波器的 x 轴扫描信号,使用示波器的 \( x\text{-}y \) 模式,可以读出两个共振吸收峰的半宽取平均值,读出扫场幅度,求出二者的比值。
      接着切换回 \( A\text{-}t \) 模式,用上面提过的方式测量出射频场频率之差\(\nu_F' - \nu_F''\),就可以利用
      \[
      \frac{1}{T_2} = \frac{\Delta\omega}{2} = \pi(\nu_F' - \nu_F'')\frac{\Delta B}{2B'}
      \]
      计算出横向弛豫时间 \( T_2 \) 的大小。



\section{结果及讨论}

  
	


\section{结论}
	在本实验测量和计算了不同散射角度下的 $\gamma$ 光子能量和相对微分散射截面, 由此验证了康普顿散射效应及其理论公式的准确性.
	实验结果显示, 散射光子的能量和微分散射截面随散射角度的变化趋势与理论预测基本一致, 但存在一定的系统性偏差.
	这可能是由于探测器的有限立体角、仪器精度有限等一系列原因. 
	本实验加深了对康普顿散射效应的理解, 未来的研究可以进一步改进实验装置, 探索更精确的散射测量方法, 并将康普顿散射应用于更多的科学和工程领域.
	
\begin{acknowledgments}
  感谢楼建玲老师的指导和帮助.
\end{acknowledgments}

% bibliography 的参数是你的 *.bib 文件去掉后缀名后的部分
\bibliography{bibli}


\end{document}
