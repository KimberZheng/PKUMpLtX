%
% PKUMpLtX --- A LaTeX document class for 'Modern Physics Laboratory' in PKU based on `revtex4-2`
%
% Please read `README.md' and the template file before using
% 需要确保 font 选项指定的字体已安装! 具体参见 `README.md' 的说明.
\documentclass[font=default]{mpltx}
\usepackage{booktabs}  
% 以下至 \begin{document} 都仅是本文件为了方便额外定义的命令, 写报告时不需要.
\hypersetup{colorlinks=true}% 超链接带颜色
\usepackage{xcolor}
\newcommand{\note}[1]{{\color{gray}#1}}
\NewDocumentCommand{\pkg}{s o m}{%
    \IfBooleanF{#1}{%
        \IfNoValueTF{#2}%
            {\href{https://www.ctan.org/pkg/#3}}%
            {\href{https://www.ctan.org/pkg/#2}}%
    }%
    {\textsf{#3}}%
}
\newcommand*\cs[1]{\texttt{\textbackslash #1}}
\newcommand*\env[1]{\textit{\texttt{#1}}}
\newcommand*\code[1]{\texttt{#1}}
\newcommand*\file[1]{\textbf{\texttt{#1}}}
\makeatletter
\newcommand\releasedate{%
    \href{https://github.com/CastleStar14654/PKUMpLtX/releases/tag/\mpltx@fileversion}%
        {\mpltx@filedate, \mpltx@fileversion}}
\makeatother
% 以上是本文件为了方便额外定义的命令, 写报告时不需要.

\begin{document}

\title{磁光克尔效应} % 切合报告内容, 简短明确, 可以不同于讲义
\author{郑熔} % 这里 \emailphone 一定要紧跟在 \author 后方
\emailphone{2300011359@stu.pku.edu.cn}{(86)19805861588}
% 如果改用 \email 则仅需要邮箱参数
\affiliation{北京大学物理学院\quad 学号: 2300011359}
% % 可以使用 \zhdate 自动生成中文日期, 如
\date{\zhdate{2025/11/26}}
% % 也可使用 babel 的 \localedate, 如
% \date{\localedate{2020}{12}{1}}
% % 两者均会输出 `2020 年 12 月 1 日'
% 下面的 \date 的参数是为了自动输出正确版本号, 正式报告请替换为上面的两种 \date 之一
% \date{\releasedate}
\begin{abstract}
  平面偏振光在光洁磁极表面发生反射时, 偏振面会出现微小偏转, 这个现象被称为磁光克尔效应. 
  它反映了物质磁化状态对其光学性质的影响. 
  本实验利用了光弹调制技术和锁相放大技术等一系列技术, 测量了一个铁磁样品薄膜 \(Pt_{73}Co_{27}\) 的克尔磁滞回线, 观察了磁光克尔效应.
  并测量了样品的饱和克尔转角\(\theta_ks = 0.196^\circ\)与矫顽力\(B = 339.432 mT\), 并探究了起偏器转角对于磁滞回线的影响. 
  这加深了我们对磁光效应的理解, 并在同时学习了解和椭偏检测相关的一些技术. 
\end{abstract}
\keywords{磁光克尔效应, 磁滞回线} % 关键词, 用逗号分隔

\maketitle
\section{引言}
1877 年,  克尔 (J.Kerr)  发现平面偏振光从光洁磁极表面反射时,  偏振面会发生微
小的偏转,  这种现象被称为克尔效应. 克尔效应与法拉第效应、 塞曼效应同样,  都是由
于物质的磁化状态改变了其光学性质所引起的现象,  被统称为磁光效应 (magneto-optical
effect)  .


磁光效应在磁畴观察、磁光存储、薄膜磁性原位表征、自旋电子学、
太阳磁场测量、原子操纵和冷却、光隔离等方面都有重要应用.由于磁光克尔效应观察
的是样品表面的反射光, 不要求样品透明, 其适用范围更宽.


磁光克尔效应就被用于观察磁性样品的磁畴结构、磁光存储技术.
近年来, 克尔效应更被用于超薄磁性膜、磁化动态过程和自旋霍尔效应研究.


本实验中, 我们通过光弹调制技术和锁相放大技术, 利用软件半自动化
进行椭偏检测, 测量了一个铁磁样品薄膜 \(Pt_{73}Co_{27}\) 的克尔磁滞回线, 观察到了磁光克
尔效应, 测量了样品的饱和克尔转角和矫顽力, 并探究了起偏器转角对于磁滞回线的影
响.从而加深了我们对磁光效应的的理解, 并在同时学习了解光弹调制、锁相放大等一系列检测技术.


\section{理论}
本实验仅考虑极入射极克尔效应\autoref{fig:yuanli}, 即磁场、入射光线均垂直与样品表面, 系统具有旋转对称性. 
\begin{figure}[htbp]
  \centering
  \includegraphics[width=0.4\linewidth]{fig/yuanli.png}
  \caption{极入射极克尔效应示意图\cite{jindaiwulishiyan}}
  \label{fig:yuanli}
\end{figure}
据菲涅尔公式, 此时反射率: 
$$
r_{\pm} = \frac{1 - n_{\pm}}{1 + n_{\pm}}
$$

$n_{\pm}$ 分别为左、右旋圆偏光的折射率. 若磁光常量 $Q$ 不为零, 入射的线偏光 (左、右旋圆偏光的叠加态) 反射后为主轴相对于入射光偏振面转过一个微小角度的椭偏光, 
一般用复克尔转角 $\tilde{\theta}_K$ 来描述: 

\[
\tan \tilde{\theta}_K = \frac{k}{r_F} = \mathrm{i}\frac{r_+ - r_-}{r_+ + r_-}.

\]
由于到 $\tilde{\theta}_K$ 一般比较小: 
\[
\tilde{\theta}_K = \theta_K + \mathrm{i}\epsilon_K \approx \tan \tilde{\theta}_K = \frac{k}{r_F} = -\mathrm{i}\frac{n_+ - n_-}{1 - n_+n_-} \approx -\mathrm{i}\frac{nQ}{1 - n^2}.
\]
其中, $\theta_K$ 和 $\epsilon_K$ 分别为复克尔转角的实部和虚部. 
也就是克尔转交和克尔椭率. 
根据琼斯矩阵等理论, 入射线偏振光在样品表面反射并经过光弹调制器和检偏器后: 
\[
\begin{bmatrix} E'_x \\ E'_y \end{bmatrix} = 
\begin{bmatrix} \frac{1}{2} & \frac{1}{2} \\ \frac{1}{2} & \frac{1}{2} \end{bmatrix} \cdot
\begin{bmatrix} 1 & 0 \\ 0 & e^{\mathrm{i}\delta} \end{bmatrix} \cdot
\begin{bmatrix} r_F & -k \\ k & r_F \end{bmatrix}
\begin{bmatrix} E_0 \\ 0 \end{bmatrix},

\]
其中 $r_F$ 表示反射光沿着入射光偏振方向 (即 $x$ 方向) 的偏振分量, $k$ 表示沿 $y$ 方向的偏振分量. 它们与复克尔转角之间的关系为: $\tan\tilde{\theta}_K = k/r_F$. 

作小量展开, 光强为: 
\[
\begin{split}
I(t) &\approx \frac{r_F^2 + k^2}{2} \left(1 + 2\theta_k \cos\delta - 2\epsilon_k \sin\delta\right) \\
&= \frac{r_F^2 + k^2}{2} \left[1 + 2\theta_k J_0\left(\delta_0\right) - 4\epsilon_k J_1\left(\delta_0\right)\sin\omega t + 4\theta_k J_2\left(\delta_0\right)\cos2\omega t + \dots\right]
\end{split}

\]

取 $\delta_0 = 2.405$ , 锁相放大器测量到的直流分量和一、二次谐波分量振幅与克尔转角、克尔椭率之间有如下关系: 
\[
\theta_K = B \frac{\sqrt{2}V_{2\omega}}{4V_0 J_2\left(\delta_0\right)}

\]
和
\[
\epsilon_K = -B \frac{\sqrt{2}V_{\omega}}{4V_0 J_1\left(\delta_0\right)},
\]
其中 $B$ 为标度系数, 是整定器对交直流信号分别进行放大导致的. 


\section{实验}
\subsection{实验装置示意}
实验装置如\autoref{fig:zhaungzhi}所示. 

\begin{figure}[htbp]
  \centering
  \includegraphics[width=0.6\linewidth]{fig/zhaungzhi.png}
  \caption{磁光克尔实验装置示意图\cite{jindaiwulishiyan}}
  \label{fig:zhaungzhi}
\end{figure}

本实验实验装置如\autoref{fig:zhaungzhi}所示, 
激光器产生的激光经过起偏器后成为线偏振光, 
近似垂直入射到样品表面;出射光经过光弹调制后进入光电探测器, 
锁相放大器接收来自光电探测器的信号并将 $V_0$、$V_\omega$ 和 $V_{2\omega}$ 传输给计算机. 
同时, 计算机还会接收光弹调制控制器的调制频率 $\omega$ 和样品处的磁感应强度 $B$. 



\subsection{实验步骤}
  \subsubsection{调节实验装置}
    启动全部实验仪器. 
    调整样品位置, 使得激光束垂直入射到样品表面, 反射光能顺利进入光弹调制器、检偏器、光电探测器.
    调整起偏器与光弹调制器平行, 检偏器与起偏器的夹角 $45^\circ$. 具体操作步骤如下: 
    \begin{enumerate}
        \item 先调节检偏器, 使光路达到几乎消光状态, 即锁相放大器的输出信号降至最小;
        \item 启动光弹调制器, 调节其振动轴的角度, 直至锁相放大器的输出信号降至最小;
        \item 最后, 将检偏器转动 $45^\circ$. 
    \end{enumerate}



  \subsubsection{克尔转角的标度}
    将起偏器小角度分别旋转\(30',1^\circ,1^\circ30',2^\circ,2^\circ30'\), 记录下此时锁相放大器的直流分量\(V_0\)和二次谐波分量\(V_{2\omega}\),
    对\(\frac{V_{2\omega}}{V_0}-\theta_k\)作线性拟合, 得到标定系数\(\alpha\)


  \subsubsection{测量克尔磁滞回线}

    
    使用测控计算机上的测控程序, 设定初始磁场、末态磁场、步长和采样时间、选择回线模式, 利用刚才标定的标度系数$\alpha$, 系统将自动采集并显示数据曲线. 


  \subsubsection{探究克尔转角和起偏器角度的关系}

    改变起偏器角度, 分别相对起偏器的刻度为\(339^\circ45'\)转动\(1^\circ,2^\circ,-1^\circ,-2^\circ\)以及将近\(-4^\circ\), 观察各个角度的磁滞回线, 探究克尔转角和起偏器角度的关系. 



\section{结果及讨论}
    按照上述实验步骤对仪器进行调节, 得到起偏器的刻度为\(339^\circ45'\),光弹调制器的角度为\(280^\circ0'\), 检偏器的角度为\(43^\circ45'\).

    克尔转角的标度系数为\(-6.84\times10^{-2}/^\circ\). 

  \subsection{测量克尔磁滞回线}

    采用起始磁场为-1000 mT, 末态磁场为1000 mT, 步长为200 mT, 对起偏器的刻度为\(339^\circ45'\)下的克尔磁滞回线进行测量, 得到克尔转角和克尔椭率. 但由于该图像的对称性较差, 对于饱和克尔转角、矫顽力的读数、以及后续的分析不太友好, 
    我们选取相对起偏器的刻度为\(339^\circ45'\)转动\(-1^\circ\)的图像进行分析, 其克尔转角和克尔椭率分别如\autoref{fig:keerzhuanjiao}和\autoref{fig:keertuolv}所示,该图对于实验中采集得到的数据点做了线性内插. 

    \begin{figure}[htbp]
      \centering
      \includegraphics[width=0.6\linewidth]{fig/new2.png}
      \caption{克尔转角磁滞回线}
      \label{fig:keerzhuanjiao}
    \end{figure}

    \begin{figure}[htbp]
      \centering
      \includegraphics[width=0.6\linewidth]{fig/new3.png}
      \caption{克尔椭率磁滞回线}
      \label{fig:keertuolv}

    \end{figure}

      观察克尔转角的磁滞回线并测量, 可以发现: 
      \begin{enumerate}
        \item 克尔转角具有较为明显的磁滞回线特征. 
        \item 饱和克尔转角为\(\theta_ks = 0.196^\circ\)
        \item 矫顽力为\(B = 339.432 mT\)
      \end{enumerate}

      观察克尔椭率的磁滞回线, 发现: 
      \begin{enumerate}
        \item  克尔椭率的磁滞回线特征并不明显. 
        \item 这是因为, 克尔椭率主要由一次谐波分量决定, 而实验装置中, 光线容易被反射、折射, 导致与我们希望测量的一次谐波分量产生干涉. 从而导致结果较为混乱. 
      \end{enumerate}

  \subsection{探究克尔转角和起偏器角度的关系}
      根据实验步骤所示多次改变起偏器的角度, 得到磁滞回线如\autoref{fig:keerzhuanjiao2}所示.

      \begin{figure}[htbp]
        \centering
        \includegraphics[width=0.6\linewidth]{fig/new1.png}
        \caption{不同起偏器角度下的克尔转角磁滞回线}
        \label{fig:keerzhuanjiao2}
      \end{figure}
      
      分析上述图像, 
      可以发现各条曲线磁滞回线的形态无显著变化, 相互之间近似为上下平移的关系. 
      读取每条磁滞回线的中心克尔转角与起偏器转角之间的关系, 得到如\autoref{fig:guifan}所示的关系图.
      \begin{figure}[htbp]
        \centering
        \includegraphics[width=0.6\linewidth]{fig/nihe.png}
        \caption{起偏器转角与中心克尔转角的关系}
        \label{fig:guifan}
      \end{figure}

      可以发现中心克尔转角与起偏器转角呈线性关系, 斜率为0.816, 与1非常接近, 验证了测量的克尔转角与起偏器夹角之间为\(\theta_k = \theta_0 + \theta_{ki}\)的关系. 
      其中, \(\theta_k\)为测量的克尔转角, \(\theta_0\)为某一常数, \(\theta_{ki}\)为样品本身的克尔转角. 

      

\section{结论}

  本次实验, 我们利用样品\(Pt_{73}Co_{27}\) 合金薄膜, 借助光弹调制器、锁相放大器等, 通过椭偏检测技术实现了对微小克尔转角的测量, 观测到了磁光克尔效应. 
  通过小范围下改变起偏器角度, 确定标定系数为\(-6.84\times10^{-2}/^\circ\). 
  通过测定样品的克尔磁滞回线, 得到了样品对于$632.8\mathrm{nm}$波长的激光的饱和克尔转角$\theta_{ks}=0.196^\circ$, 矫顽力为$B_c=339.432mT$. 
  证明了样品具有铁磁性. 
  同时研究了起偏器角度对于克尔转角的影响, 验证了测量的克尔转角与起偏器夹角之间为线性关系. 

\begin{acknowledgments}
  感谢搭档吴浅溪和周路群老师的指导和帮助.
\end{acknowledgments}

% bibliography 的参数是你的 *.bib 文件去掉后缀名后的部分
\bibliography{bibli}

\appendix

\section{思考题}
\subsection{我们的实验装置对克尔转角和克尔椭偏率的测量精度是否一样高?为什么?}
  不一样高.克尔转角正比于二次谐波分量, 克尔椭偏率正比于一次谐波分量.
  光束在光弹调制晶体的两个表面和其他地方发生多次折射、反射, 导致光
  束干涉现象, 影响一次谐波的测量, 所以对克尔椭偏率的测量是不准确的.

\subsection{如果用一个以角速度$\omega$旋转的$\frac{\lambda}{2}$波片代替光弹调制器, 光电探测器的输出信号会如何变化, 是否也能测出复克尔转角?}
以角速度$\omega$转动的$\frac{\lambda}{2}$波片, 其琼斯矩阵为: 
\[
\begin{bmatrix}
\cos\omega t & -\sin\omega t \\
\sin\omega t & \cos\omega t
\end{bmatrix}
\cdot
\begin{bmatrix}
1 & 0 \\
0 & -1
\end{bmatrix}
\cdot
\begin{bmatrix}
\cos\omega t & \sin\omega t \\
-\sin\omega t & \cos\omega t
\end{bmatrix}
=
\begin{bmatrix}
\cos2\omega t & \sin2\omega t \\
\sin2\omega t & -\cos2\omega t
\end{bmatrix}
.
\]


\[
\begin{bmatrix}
\frac{1}{2} & \frac{1}{2} \\
\frac{1}{2} & \frac{1}{2}
\end{bmatrix}
\begin{bmatrix}
\cos2\omega t & \sin2\omega t \\
\sin2\omega t & -\cos2\omega t
\end{bmatrix}
\begin{bmatrix}
r_F \\
k
\end{bmatrix}
=
\frac{1}{2}
\begin{bmatrix}
(\sin2\omega t + \cos2\omega t)r_F + (\sin2\omega t - \cos2\omega t)k \\
(\sin2\omega t + \cos2\omega t)r_F + (\sin2\omega t - \cos2\omega t)k
\end{bmatrix}
\]

光强为: 
\[
I(t) = r_F^2 + k^2 + r_F^2 \sin4\omega t - k^2 \sin4\omega t + r_F k (sin 4 \omega t - 1).
\]


由此可见,  出射光的强度包含直流分量和四次谐波分量,  而四次谐波分量中耦合信息, 
因此无法实现对复克尔转角的实部和
虚部的有效提取,  因此不能测出复克尔转角. 
\end{document}
