%
% PKUMpLtX --- A LaTeX document class for 'Modern Physics Laboratory' in PKU based on `revtex4-2`
%
% Please read `README.md' and the template file before using
% 需要确保 font 选项指定的字体已安装! 具体参见 `README.md' 的说明.
\documentclass[font=default]{mpltx}
\usepackage{booktabs}  
% 以下至 \begin{document} 都仅是本文件为了方便额外定义的命令, 写报告时不需要.
\hypersetup{colorlinks=true}% 超链接带颜色
\usepackage{xcolor}
\newcommand{\note}[1]{{\color{gray}#1}}
\NewDocumentCommand{\pkg}{s o m}{%
    \IfBooleanF{#1}{%
        \IfNoValueTF{#2}%
            {\href{https://www.ctan.org/pkg/#3}}%
            {\href{https://www.ctan.org/pkg/#2}}%
    }%
    {\textsf{#3}}%
}
\newcommand*\cs[1]{\texttt{\textbackslash #1}}
\newcommand*\env[1]{\textit{\texttt{#1}}}
\newcommand*\code[1]{\texttt{#1}}
\newcommand*\file[1]{\textbf{\texttt{#1}}}
\makeatletter
\newcommand\releasedate{%
    \href{https://github.com/CastleStar14654/PKUMpLtX/releases/tag/\mpltx@fileversion}%
        {\mpltx@filedate, \mpltx@fileversion}}
\makeatother
% 以上是本文件为了方便额外定义的命令, 写报告时不需要.

\begin{document}

\title{核磁共振} % 切合报告内容, 简短明确, 可以不同于讲义
\author{郑熔} % 这里 \emailphone 一定要紧跟在 \author 后方
\emailphone{2300011359@stu.pku.edu.cn}{(86)19805861588}
% 如果改用 \email 则仅需要邮箱参数
\affiliation{北京大学物理学院\quad 学号: 2300011359}
% % 可以使用 \zhdate 自动生成中文日期, 如
\date{\zhdate{2025/11/12}}
% % 也可使用 babel 的 \localedate, 如
% \date{\localedate{2020}{12}{1}}
% % 两者均会输出 `2020 年 12 月 1 日'
% 下面的 \date 的参数是为了自动输出正确版本号, 正式报告请替换为上面的两种 \date 之一
% \date{\releasedate}
\begin{abstract}
  核磁共振是具有非零自旋的原子核在外磁场作用下, 
  吸收特定频率电磁波发生能级跃迁的物理过程. 
  本实验从共振频率的基础调节操作入手, 
  首先观察了掺有三氯化铁的水样品的核磁共振基本现象, 探究射频场频率、幅度、扫场幅度以及电路盒左右位置对共振信号的波形、幅度和位置分布的影响. 
  随后, 利用共振条件以及水样品中质子的共振频率对永磁铁的磁场强度进行校准, 并基于校准值测量聚四氟乙烯样品中氟原子核的 g 因子, 
  同时测定氟核的横向弛豫时间. 
\end{abstract}
\keywords{核磁共振, g 因子, 弛豫时间} % 关键词, 用逗号分隔

\maketitle
\section{引言}
Pauli于1924年研究元素光谱精细结构时首次提出核磁矩与核自旋概念. 
1939年, 经I.I.Rabi改进的Otto Stern首创分子束实验, 
提出更精确的核磁矩测量方法. 
1946年, 美国Harvard大学的Purcell、Pound与Stanford大学的Bloch、Hanson同时独立设计近代核磁共振技术. 
在核物理领域, 通过测量各种核磁矩大小, 可获取诸多核结构信息. 
本实验探究了核磁共振实验现象, 并利用核磁共振校准磁场、 测量氟核的\( g \)因子以及横向弛豫时间. 

\section{理论}
根据共振时 \( h\nu = \Delta E \) 和\( \Delta E = \gamma \hbar B\)可以把共振条件写成


\[
\omega = \gamma B \quad \text{或} \quad \nu = \frac{\gamma}{2\pi} B 
\]


式中 \( \omega = 2\pi\nu \) 为射频场的圆频率.
\( \gamma/2\pi \) 称为原子核的回旋频率,数值上等于磁场为1 T时的磁共振频率,
单位为 MHz/T.若已知磁场 \( B \),测量共振频率可求出 \( \gamma/2\pi \) 的数值.反之,可利用\( \gamma/2\pi \) 为已知值来校准磁场. 



热平衡条件下之所以可以观测到核磁共振现象, 是因为体系上下能级的粒子数分布遵循玻尔兹曼分布: 


\[
\frac{N_{20}}{N_{10}} = \exp(-\Delta E/kT)
\]


其中 \( N_{20} \)、\( N_{10} \) 分别是上、下能级的粒子数. 一般情形下, \( \Delta E \ll kT \), 近似有


\[
\frac{N_{20}}{N_{10}} \approx 1 - \frac{\Delta E}{kT} = 1 - \frac{\gamma \hbar B}{kT}
\]


由此可求出热平衡时下能级与上能级的粒子数之差为


\[
n_0 = N_{10} - N_{20} \approx \frac{\gamma \hbar B}{2kT}N \tag{6-1-4}
\]


其中 \( N \) 为粒子总数, 对于氢核, 在室温下当磁场为 \( 1 \, \text{T} \) 时


\[
n_0 \approx 0.000\ 003\ 4\ N
\]


这个粒子数差提供了观察核磁共振的可能性. 
磁场 \( B \) 越强, 粒子差数越大, 对观察核磁共振信号越有利;而温度越高, 粒子差数越小, 对观察核磁共振信号越不利. 

粒子的横向弛豫时间由以下公式决定: 
\[
\frac{1}{T_2} = \frac{\Delta \omega}{2} 
\]



其中 \( \Delta \omega \) 为共振峰的半高宽. 


示波器产生的波形与扫场速度、横向弛豫时间有关\autoref{fig:saochang}. 扫场速度越快, 尾波数量越多. 外磁场越均匀, 横向弛豫时间越长, 尾波数量越多. 以此可以判断外磁场均匀性. 

\begin{figure}[htbp]
  \centering
  \includegraphics[width=0.6\linewidth]{fig/saochang.png}
  \caption{扫场速度不同时所观测得到的示波器信号波形\cite{jindaiwulishiyan}}
  \label{fig:saochang}
\end{figure}

\section{实验}
\subsection{实验装置示意}
实验装置如\autoref{fig:zhaungzhi}所示. 

\begin{figure}[htbp]
  \centering
  \includegraphics[width=0.6\linewidth]{fig/shiyanzhaungzhi.png}
  \caption{核磁共振实验装置示意图\cite{jindaiwulishiyan}}
  \label{fig:zhaungzhi}
\end{figure}


\subsection{实验步骤}
  \subsubsection{观察掺有三氯化铁水样品的核磁共振现象, 确定均匀场的位置}
    将水样品 (含三氯化铁) 放置在核磁共振仪中, 
    将扫场线圈的幅度置于 100 格左右, 将 “检波输出” 端和示波器连接, 
    将 “频率测试” 端和数字频率计连接, 调节示波器内扫描为 5ms / 格. 
    连续调节边限振荡器的频率, 并观察示波器波形, 直到可以观察到共振吸收峰. 
    之后依次改变 \textbf {射频场频率}、\textbf {扫场幅度}、\textbf {电路盒左右位置} 和 \textbf {射频场幅度}, 
    观察对吸收信号的波形、幅度以及位置分布的影响. 


    调整电路盒的左右位置, 同时观察示波器上的波形, 当波形中尾波最多时, 记录下此时电路盒的位置, 这时候外磁场的均匀性最好. 

  \subsubsection{校准永磁铁的磁场}
    将样品置于前面中所测量的磁场均匀性最好的位置, 
    找到共振吸收信号后尽可能减小扫场幅度, 这样可以减小共振频率测量误差. 
    调整射频场频率使得示波器上出现多个共振峰, 峰与峰之间间隔为 10ms 左右, 记录下此时的射频场频率\( \nu_H \). 
    根据共振条件和\( 25^\circ\text{C} \)水样品中质子的回旋频率\( \frac{\gamma}{2\pi} \), 
    可以求出磁场的大小
    \[
    B_0 = \frac{\nu_H}{\gamma/2\pi}
    \]

    为了估计频率测量的精度, 可以调节射频场频率使得示波器每两个共振峰合并为1个, 即峰与峰之间间隔变为 20ms 左右, 记录下满足该情形的频率分别为 \( \nu_H' \) 和 \( \nu_H'' \) ( \( \nu_H' < \nu_H'' \)) . 
    由此对频率测量的不确定度进行估计
    \[
    \Delta B = \frac{B'}{10} = \frac{(\nu_H' - \nu_H'')/20}{\gamma/2\pi}
    \]
    从而可以得到磁场的不确定度. 

    \subsection{测量氟核的 \( g \) 因子}

      将聚四氟乙烯样品盒放在与2中相同的位置(均匀磁场处), 调节共振后利用
      \[
      g = \frac{\nu_F / B_0}{\mu_N / h}
      \]
      计算氟核的 \( g \) 因子, 可以利用和2中相同的方法测量出频率的精度, 并结合磁场的精度对结果的不确定度进行估计. 

    \subsection{测量氟核的横向弛豫时间}

      将“扫场输出”信号作为示波器的 x 轴扫描信号, 使用示波器的 \( x\text{-}y \) 模式, 可以读出两个共振吸收峰的半宽取平均值, 读出扫场幅度, 求出二者的比值. 
      接着切换回 \( A\text{-}t \) 模式, 用上面提过的方式测量出射频场频率之差\(\nu_F' - \nu_F''\), 就可以利用
      \[
      \frac{1}{T_2} = \frac{\Delta\omega}{2} = \pi(\nu_F' - \nu_F'')\frac{\Delta B}{2B'}
      \]
      计算出横向弛豫时间 \( T_2 \) 的大小. 



\section{结果及讨论}

  \subsection{观察掺有三氯化铁水样品的核磁共振现象}
    调节出共振信号后继续调节射频场的频率,  直到共振信号在示波器上呈现 10ms 的等间距分布. 在共振峰峰值最大, 尾波数量最多的地方, 磁场均匀并达到共振条件. 
    记录下此时的射频场频率\( \nu_H = 21.10676 \, \text{MHz} \), 电路盒右边缘对应的位置\(x = 0.70cm\). 

    \begin{enumerate}
      \item 改变射频场频率: 当频率偏离共振频率时, 相邻共振峰两两相互接近, 峰值高度减小, 直到两个共振峰合并为一个, 此时相邻共振峰的距离为 20 ms , 再沿着刚才调节的方向调节, 共振峰消失. 
      \item 改变扫场幅度: 当幅度偏离最高幅度时, 也会发生和调整射频场幅度类似的现象. 相邻共振峰两两相互接近, 合并为一个, 再调节, 共振峰消失. 
      \item 改变电路盒左右位置: 当电路盒位置偏离均匀磁场位置时, 共振峰的高度降低, 尾波数量减少, 最终消失. 
      \item 改变射频场幅度: 对示波器图像的影响较小. 
    \end{enumerate}
	
    产生上述现象的原因如下, 示意图如\autoref{fig:yuanli}所示. 

    \begin{figure}[htbp]
      \centering
      \includegraphics[width=0.6\linewidth]{fig/yuanli.png}
      \caption{扫场、静磁场与共振信号的关系\cite{jiaoda}}
      \label{fig:yuanli}
    \end{figure}

    外加总磁场为\( B = B_0 \pm B' \cos \omega t \), 其中前者为扫场, 后者为射频场. 

    \begin{enumerate}

      \item 若固定扫场幅度, 调节射频场频率, 即改变\autoref{fig:yuanli}中\(2\pi\nu/\gamma\)和总磁场 B 的交点, 
      则会改变两个共振峰之间的距离, 从而使得示波器上相邻共振峰两两接近, 最终合并为一个. 

      \item 若固定射频场频率, 调节扫场幅度, 即改变\autoref{fig:yuanli}中\( B_0 \pm B'  \cos \omega t \)的位置. 随着\(B_0\)的不断增大, 
      \(2\pi\nu/\gamma\)和总磁场 B 的交点从无到有, 相邻两交点的逐渐分开, 再和另一个相邻的交点接近, 最终再合并. 

      \item 改变电路盒位置, 相当于改变了\( B_0 \)的均匀性, 而不均匀的磁场会减小样品的横向弛豫时间, 导致尾波衰减更快, 从而影响了共振峰的高度和尾波数量. 经过观察, 磁场最均匀的地方在电路盒右边缘位置\(x = 0.70cm\)处. 

      \item 改变射频场幅度, 对示波器图像的影响较小, 因为改变射频场幅度对峰间距影响不大, 而只会影响峰的高度. 
    
    \end{enumerate}


  \subsection{校准永磁铁的磁场}
    将样品置于磁场均匀性最好的位置\(x = 0.70cm\), 找到共振信号后尽可能减小扫场幅度, 这样可以减小共振频率测量误差. 
    减小扫场幅度为15.2格, 调整射频场频率使得示波器上出现多个共振峰, 峰与峰之间间隔为 10ms 左右, 记录下此时的射频场频率\( \nu_H \). 
    根据共振条件和\( 25^\circ\text{C} \)水样品中质子的回旋频率\( \frac{\gamma}{2\pi} = 42.5763888 \, \text{MHz/T} \), 

    为了估计频率测量的精度, 可以调节射频场频率使得示波器每两个共振峰合并为1个, 即峰与峰之间间隔变为 20ms 左右, 
    记录下满足该情形的频率分别为 \( \nu'\) 和 \( \nu''\). 由此得到\autoref{table:f}.

    \begin{table}[htbp]
      \centering
      \begin{tabular}{ccc}
          \hline
          $\nu / \mathrm{MHz}$ & $\nu' / \mathrm{MHz}$ & $\nu'' / \mathrm{MHz}$ \\
          \hline
          21.10232 & 21.08453 & 21.11984 \\
          \hline
      \end{tabular}
      \label{table:f}
    \end{table} 

    $$\nu = \frac{\gamma}{2\pi} B \Rightarrow B = \frac{2\pi \nu}{\gamma} = 0.495739\ \mathrm{T}$$

    $$B' = \frac{(\nu' - \nu'')/2}{(\gamma/2\pi)} = 4.1467 \times 10^{-4}\ \mathrm{T}$$

    $$\Delta B = \frac{B'}{10} = 4.1467 \times 10^{-5}\ \mathrm{T}$$

    对于永磁铁的磁场校准的最终结果为: 

    $$B_0 = 0.49574 \pm 0.00004 \ \mathrm{T}$$


  \subsection{测量氟核的 \( g \) 因子}
    将聚四氟乙烯样品盒放在与2中相同的位置(均匀磁场处), 扫场幅度调节至25格. 利用上述相同方法, 测量得到频率如\autoref{table:g}. 
    \begin{table}[htbp]
      \centering
      \begin{tabular}{ccc}
          \hline
          $\nu / \mathrm{MHz}$ & $\nu' / \mathrm{MHz}$ & $\nu'' / \mathrm{MHz}$ \\
          \hline
          19.84755 & 19.82411 & 19.87056 \\
          \hline
      \end{tabular}
      \label{table:g}
    \end{table} 

    根据共振条件, 计算氟核的 \( g \) 因子: 
    $$  
      g = \frac{\nu_0 / B_0}{\mu_N / h} = 5.252308 
    $$ 


    其中氟核的磁矩与核自旋比值 \( \frac{\mu_N}{h} = 7.62259396\ \mathrm{MHz/T} \). 
    结合频率和磁场的精度, 计算 \( g \) 因子的不确定度: 
    

    $$
      \Delta \nu = \frac{\nu' - \nu''}{20} = 2.3225 \times 10^{-3}\ \mathrm{MHz} 
    $$
    $$  
      \frac{\Delta g}{g} = \sqrt{\left( \frac{\Delta \nu}{\nu} \right)^2 + \left( \frac{\Delta B_0}{B_0} \right)^2} = 1.4 \times 10^{-4} 
    $$
    $$  
    \Delta g = 0.0007 
    $$
    因此氟核的 \( g \) 因子为: 
    $$
      g = 5.2523 \pm 0.0007
    $$

  \subsection{测量氟核的横向弛豫时间}
    将“扫场输出”信号作为示波器的 x 轴扫描信号, 使用示波器的 \( x\text{-}y \) 模式, 
    可以读出两个共振吸收峰的半宽分别为0.182和0.170, 取其平均值为\(\Delta B = 0.176\), 
    读出扫场变化范围为 \(2B' = 6.49\) , 求出二者的比值为\(k = \Delta B / 2B' = 0.0271186\). 

    接着切换回 \( A\text{-}t \) 模式, 用上面提过的方式测量出频率如\autoref{table:t}. 
    \begin{table}[htbp]
      \centering
      \begin{tabular}{ccc}
          \hline
          $\nu / \mathrm{MHz}$ & $\nu' / \mathrm{MHz}$ & $\nu'' / \mathrm{MHz}$ \\
          \hline
          19.82649 & 19.66635 & 19.99060 \\
          \hline
      \end{tabular}
      \label{table:t}
    \end{table} 

    由此可以利用
    $$
      \frac{1}{T_2} = \frac{\Delta\omega}{2} = \pi(\nu' - \nu'')\frac{\Delta B}{2B'}
    $$

    计算得到$\frac{1}{T_2} = 0.027625 \mathrm{MHz}$, 从而得到横向弛豫时间为$ T_2 = 36.1995 \mathrm{ \mu s}$.

  \subsection{观察纯水中的核磁共振现象}

  由于实验室台面上似乎并没有看到纯水样品, 这里推测理论上纯水中的核磁共振现象:
  相对掺有三氯化铁的水样品而言, 纯水样品的共振信号极弱, 在同等实验条件下难以观察到明显的共振峰, 
  且背景噪声对信号的影响很大, 共振信号的波形和幅度会受到较大干扰. 

  产生这种现象的原因见附录中思考题的回答.



\section{结论}
  本实验从核磁共振现象的基本原理出发, 完成了对核磁共振现象的调节和观测. 
  在实验过程中, 改变了射频场频率、扫场幅度、电路盒左右位置和射频场幅度等和核磁共振相关的参量, 
  观察了这些改变对于共振信号的影响, 并分析了这些现象产生的原因. 
  实验还利用了核磁共振完成了对永磁铁的磁场均匀性和磁场强度的校准, 得到本实验中磁场强度为
  $B_0 = 0.49574 \pm 0.00004 \ \mathrm{T}$, 
  随后测量了聚四氟乙烯样品中氟核的 g 因子,  得到结果为
  $g = 5.2523 \pm 0.0007$,  最后对氟核的横向弛豫时间进行了测量, 得到结果为$ T_2 = 36.1995 \ \mu s$.
\begin{acknowledgments}
  感谢季伟老师的指导和帮助.
\end{acknowledgments}

% bibliography 的参数是你的 *.bib 文件去掉后缀名后的部分
\bibliography{bibli}

\appendix

\section{思考题}
\subsection{对 “纯水” 样品与掺有三氯化铁的 “水” 样品中质子共振信号的主要差别作出解释}
掺有三氯化铁的水样品中, 三氯化铁作为一种顺磁性物质,
能够显著缩短质子的弛豫时间, 提高核磁共振信号的强度和清晰度.
纯水样品中, 质子的弛豫时间较长, 导致信号较弱且难以分辨,
因此在实验中更难观察到明显的共振峰.

\end{document}
